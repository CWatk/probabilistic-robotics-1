
\documentclass[10pt]{article}
\usepackage{amsmath}
\begin{document}
\title{Solutions for Chapter 3 of Probabilistic Robotics}

\section{Problem 1}
\begin{enumerate}
  \item \textit{In this and the following exercises, you are asked to design a Kalman
filter for a simple dynamical system: a car with linear dynamics moving in a
linear environment. Assume $\delta t = 1$ for simplicity. The position of the
car at time $t$ is given by $x_t$. Its velocity is given by $\dot{x}_t$, and its
acceleration is given by $\ddot{x}_t$. Suppose the acceleration is set
randomly at each point in time, according to a Gaussian with zero mean and
covariance $\sigma^2 = 1$.}

One thing I note: At first it sounded to me like we would know what the
acceleration value was at each time step. This threw me off for a long time,
until I realized that this is more like modeling a random wind blowing - you
don't know what the effect is at any given moment, all you know is it's
statistical behavior. Since we don't \textit{know} the value of the
acceleration, it \textit{can't} be part of our state! That may be obvious to
others reading the problem, but I worked on it for two days (!) before I figured
it out. C'est la vie.

  \begin{enumerate}
  \item \textit{What is a minimal state vector for the Kalman filter (so that the
    resulting system is Markovian)?} 

    In order to be Markovian, we have to have a state vector such that the
    future and the past are independent given the present state. If the state
    vector has the position and velocity, this condition is met. All of the
    acceleration values of the past are completely summarized in the position
    and velocity, so keeping it as a state variable doesn't tell us anything new
    in terms of predicting the future. The acceleration is set randomly at each
    time step, but given that we know the position and velocity we don't need to
    know any past acceleration in order to compute the future given the state
    and the control (which in this case will be setting the acceleration).

    Our state vector is thus:
  $$\begin{pmatrix}x_t \\ \dot{x}_t\end{pmatrix}$$

  \item \textit{For your state vector, design the state transition probability 
    $p(x_t | u_t,x_{t-1})$. Hint: this transition function will possess linear
  matrices $A$ and $B$ and a noise covariance $R$.}

  We will use the moments parameterization since we're building a Kalman filter.
  The state transition function is of the form
  $$x_t = A x_{t-1} + B u_t + \epsilon_t$$

  Where $\epsilon_t$ is a gaussian random variable with mean zero and variance 1
  (the acceleration, as given in the problem). We'll assume there is no
  acceleration, and incorporate the acceleration as error in the position and
  velocity.
 
  The mean for our distribution is given by $A x_{t-1} + B u_t$, and the
  variance is given by $A\Sigma_{t-1}A^T + cov(\epsilon_t)$. The
  $A\Sigma_{t-1}A^T$ term carries forward the uncertainty from the previous state
  updated by the state transition function (i.e. we need to apply the same
  transformation to our error bounds that we apply to the estimate, otherwise
  the error bounds will become meaningless). The $cov(\epsilon_t)$ term is the
  new error added in by the random acceleration.

  All that remains is to use the equations of motion to derive matrices for $A$
  and $B$. We know that:
  \begin{gather}
    x_t = x_{t-1} + \dot{x}_{t-1} \Delta t \\ 
    \dot{x}_t = \dot{x}_{t-1} 
  \end{gather}

  Note that we're ignoring acceleration in the above expressions, because we're
  going to incorporate the acceleration via the error term.

  There is no control action in this model, since there is no explicit (known)
  acceleration (i.e. the acceleration value is just a probability distribution
  and we don't know its value at any given time). That means $B u_t = \vec{0}$.

  Now we can write our prediction of the overall mean, in matrix form:

  $$
\overline{\mu}_t = \begin{pmatrix}x_t \\ \dot{x}_t\end{pmatrix} = 
\begin{pmatrix}1 & \Delta t \\ 0 & 1\end{pmatrix} \begin{pmatrix}x_{t-1} \\
\dot{x}_{t-1} \end{pmatrix}
  $$

  Next we need to model the covariance. Since we are given the variance of the
  acceleration (=1), we have to figure out how that maps into variance of the
  two state variables and make a variance vector accordingly. The full equations
  of motion would look like this:

  \begin{gather}
    x_t = x_{t-1} + \left(\frac{\dot{x}_{t-1} + \dot{x}_t}{2}\right)\Delta_t\\
    \dot{x}_t = \dot{x}_{t-1} + \ddot{x}_{t-1}\Delta_t
  \end{gather}

  We can eliminate $\dot{x}_t$ from the $x_t$ term by substituting in our
  expression for $\dot{x}_t$:

  \begin{gather}
    x_t = x_{t-1} + \left(\dot{x}_{t-1} + \frac{\ddot{x}_{t-1}}{2}\right)\Delta_t
  \end{gather}

  That gives us what we need to figure out how the random acceleration should
  affect our estimates for position and velocity. The variance of velocity
  should be equal to the variance of acceleration times the time elapsed, and
  the variance of position should be equal to half of that. 

  \begin{gather}
  cov(\delta_t) = \begin{pmatrix}\frac{1}{2} & 0 \\ 0 & 1\end{pmatrix}\Delta_t
  \end{gather}

  Now we have our probability $p(x_t | u_t,x_{t-1})$:
  \begin{gather}
  \overline{\mu}_t = \begin{pmatrix}1 & \Delta_t \\ 0 & 1\end{pmatrix}
  \begin{pmatrix}x_{t-1} \\ \dot{x}_{t-1}\end{pmatrix}\\
  \overline{\Sigma}_t = \begin{pmatrix}1 & \Delta_t \\ 0 & 1\end{pmatrix}
    \Sigma_{t-1}
  \begin{pmatrix}1 & 0 \\ \Delta_t & 1\end{pmatrix} + 
  \begin{pmatrix}\frac{\Delta_t}{2} & 0 \\ 0 & \Delta_t\end{pmatrix}
  \end{gather}

  We have to provide an initial value for $\Sigma_0$. In this case, I think it's
  reasonable to assume that it is $0$. The uncertainty from the random
  accelerations at each timestep will propagate into $\Sigma_t$, so it will not
  remain $0$ for long.

  Later, the Kalman gain will be used to select how much to weight the
  prediction versus the correction (measurement), based on the relative
  magnitudes of the covariances.

\item \textit{Implement the state prediction step of the Kalman filter. Assuming
    we know at time $t=x, x_0 = \dot{x}_0 = \ddot{x}_0 = 0$. Compute the state
  distributions for times $t=1,2,...,5$.}

\item \textit{For each value of $t$, plot the joint posterior over $x$ and
    $\dot{x}$ in a diagram, where x is the horizontal and $\dot{x}$ is the
    vertical axis. For each posterior, your are asked to plot an uncertainty
    ellipse, which is the ellipse of points that are one standard deviation away from
    the mean. Hint: if you do not have access to a mathematics library, you can
  create those ellipses by analyzing the eigenvalues of the covariance matrix.}

\item \textit{What will happen to the correlation between $x_t$ and $\dot{x}_t$
  as $t\to \infty$?}
  \end{enumerate}

\end{enumerate}

\end{document}

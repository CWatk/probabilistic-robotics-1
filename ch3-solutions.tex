
\documentclass[10pt]{article}
\usepackage{amsmath}
\begin{document}
\title{Solutions for Chapter 3 of Probabilistic Robotics}

\section{Problem 1}
\begin{enumerate}
  \item In this and the following exercises, you are aksed to design a Kalman
filter for a simple dynamical system: a car with linear dynamics moving in a
linear environment. Assume $\delta t = 1$ for simplicity. The position of the
car at time $t$ is given by $x_t$. Its velocity is given by $\dot{x}_t$, and its
acceleration is given by $\ddot{x}_t$. Suppose the acceleration is set
randomly at each point in time, according to a Gaussian with zero mean and
covariance $\sigma^2 = 1$.

  \begin{enumerate}
  \item \textit{What is a minimal state vector for the Kalman filter (so that the
resulting system is Markovian)?} 

At first glance it's tempting to think we just need to have the acceleration,
because the velocity and position can be derived from that.  However, we need to
keep track of the velocity and position as well because otherwise we'd have to
keep the entire history of acceleration along with the initial conditions in
order to recover the constants of integration for each quantity (i.e. the state
would not satisfy the Markov assumption). So our state vector should be
$$ \begin{pmatrix} x_t \\ \dot{x}_t \\ \ddot{x}_t \end{pmatrix} $$

  \item \textit{For your state vector, design the state transition probability 
  $p(x_t | u_t,x_{t-1})$. Hint: this transition function will possess linear
  matrices $A$ and $B$ and a noise covariance $R$.}

  The state transition \textit{function} is going to be the basic equations of
  motion for each term in the state vector:

  \begin{math}
  x_t = x_{t-1} + \dot{x}_t \Delta t \\ 
  \dot{x}_t = \dot{x}_{t-1} + \ddot{x}_t \Delta t \\ 
  \ddot{x}_t = 0\\ 
  \end{math}
  The acceleration will be altered by the control action and noise term. 1 is
  the mean, so the control should be to set the acceleration to 1. This
  portion of the update is just concerned with non-control stuff (momentum and
  position).

  Switching to matrix notation:

  $$
  \begin{pmatrix} x_t \\ \dot{x}_t \\ \ddot{x}_t \end{pmatrix} = 
  \begin{pmatrix} 1 & \Delta t & 0 \\ 0 & 1 & \Delta t  \\ 0 & 0 & 0 \end{pmatrix} 
  \begin{pmatrix} x_{t-1} \\ \dot{x}_{t-1} \\ \ddot{x}_{t-1} \end{pmatrix} + B u_t + \delta t
  $$

  Now for the control action, we only set the acceleration so it's
  one-dimensional. In order to make it add to the state vector with the
  appropriate result, we need to make a 3x1 matrix and multiply it by the
  acceleration value (which has mean 1):

  $$ B u_t = \begin{pmatrix} 0 \\ 0 \\ 1 \end{pmatrix} 1$$

  Since we're using the moments parameterization of the gaussian distribution,
  this gives us our prediction mean:

  $$ 
  \overline{\mu}_t = \begin{pmatrix} 1 & \Delta t & 0 \\ 0 & 1 & \Delta t  \\ 0 & 0 & 0 \end{pmatrix}
  \begin{pmatrix} x_{t-1} \\ \dot{x}_{t-1} \\ \ddot{x}_{t-1} \end{pmatrix} + 
  \begin{pmatrix} 0 \\ 0 \\ 1 \end{pmatrix} 1
  $$

  Finally, we need to add in the error term to capture the uncertainty. We know
  that it's gaussian with mean 1, but we accounted for the mean in the control
  update term so we'll use a mean of zero for the error variance here. The
  expression we're going to create is a probability distribution, meaning that
  it has to be a function of the state. Since the state is vector-valued, and
  since the probability is only specified for the acceleration, we can select
  the acceleration term by designing an appropriate matrix for the covariance.

  The multivariate gaussian is defined as: 
  $$ p(x) = det(2\pi \Sigma)^{-\frac{1}{2}} e^{-\frac{1}{2} (x - \mu)^T
  \Sigma^{-1} (x - \mu)} $$

\end{enumerate}

\end{enumerate}

\end{document}

\documentclass{article}
\usepackage{amsmath}
\author{Erik Lee}
\date{2014-10-24}
\title{A digression on the mysteries of covariance matrices}
\begin{document}
\maketitle

One thing that I'm struggling with is getting a solid grasp of how to make a
covariance matrix out of a transition vector and a variance. Taking the example
from the book (Probabilistic Robotics, ch3, ex 1), we have a linear system
undergoing random, unpredictable accelerations. Our state vector is
$\left(x \dot{x}\right)^T$, and the acceleration ($\ddot{x}$) is known to have
mean zero and variance 1.

\begin{gather}
  x_t = x_{t-1} + \dot{x}_t-1\Delta_t + \frac{1}{2}\ddot{x}_t\Delta_t^2 \\
  \dot{x}_t = \dot{x}_{t-1} + \ddot{x}_t\Delta_t 
\end{gather}

See the chapter 3 solutions for details about how I solved this, my concern
here is trying to figure out why the covariance matrix should be what it is.

To start with, we know that the variance of the acceleration
($\sigma_{\ddot{x}_t}^2$) is 1. We know that the acceleration should contribute
to the position and velocity according to these equations:

\begin{gather}
  \Delta x_a = a \frac{\Delta t^2}{2} \\
  \Delta \dot{x}_a = a \Delta t 
\end{gather}

What we want to do is figure out what the effect of a given variance on
acceleration would be in terms of $x$ and $\dot{x}$. Assume that there aren't
any other unknown influences on the state variables, so that \textit{all} of
their respective uncertainty arises from uncertainty in the acceleration. Then
the standard deviation of the postion ($x$) should be just the standard
deviation of the acceleration times the amount that the acceleration contributes
to the calculation of $x$. For example, if the standard deviation of the
acceleration is 1 $m/s^2$, then over the course of $dt$ seconds, the standard
deviation of $x$ should increase by $\frac{dt^2}/{2}$, and the standard
deviation of the velocity should increase by $dt$. Now here's where it gets
tricky - if we \textit{already} have some uncertainty in the velocity, and we
then add \textit{more} uncertainty via the acceleration, then \textit{both} of
those uncertainties have to combine to create the total uncertainty in the
position. Likewise, the prior uncertainty in the velocity has to contribute to
the new uncertainty along with the added uncertainty from the acceleration.
That's where the covariance matrix comes into play. The total uncertainty in $x$
is going to be the sum of the uncertainty in the velocity and the undertainty in
the acceleration, with their respective scaling factors applied.


\end{document}

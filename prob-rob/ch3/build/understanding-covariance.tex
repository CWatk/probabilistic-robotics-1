\documentclass{article}
\author{Erik Lee}
\date{2014-10-24}
\title{A digression on the mysteries of covariance matrices}
\maketitle
\usepackage{amsmath}
\begin{document}

One thing that I'm struggling with is getting a solid grasp of how to make a
covariance matrix out of a transition vector and a variance. Taking the example
from the book (Probabilistic Robotics, ch3, ex 1), we have a linear system
undergoing random, unpredictable accelerations. Our state vector is
$\left(x,\dot{x}\right)$, and the acceleration ($\ddot{x}$) is known to have
mean zero and variance 1.

\begin{gather}
x_t = x_{t-1} + \dot{x}_t-1\Delta_t + \frac{1}{2}\ddot{x}_t\Delta_t^2
\dot{x}_t = \dot{x}_{t-1} + \ddot{x}_t\Delta_t 
\end{gather}

See the chapter 3 solutions for details about how I solved this, my concern
here is trying to figure out why the covariance matrix should be what it is.

To start with, we know that the variance of the acceleration
($\sigma_{\ddot{x}_t}^2$) is 1. We know that the acceleration should contribute
to the position and velocity according to these equations:

\begin{gather}
  \Delta x_a = a \frac{\Delta t^2}{2} \\
  \Delta \dot{x}_a = a \Delta t 
\end{gather}

\end{document}
